\section{Linear Approximation of Virtual Pivot Point Model of 2D Spoked Runner}

Extended from the vertical hopper, this model is aimed to use for analysis of coupled dynamics of the spoked runner, which has following assumptions

\begin{itemize}
\item massless leg
\item pointmass as the body
\end{itemize}


\subsection{System Kinematics}
As indicated in Fig XXX, the position of the body (mass) is
\begin{align*}
x &= lcos\theta + r_ccos(\theta +\phi)\\
z &= lsin\theta + r_csin(\theta +\phi)\\
\end{align*}
and the velocity
\begin{align*}
\dot x &= \dot lcos\theta -lsin\theta\dot \theta -  r_csin(\theta +\phi)(\dot\theta +\dot\phi)\\
\dot z &= \dot lsin\theta + lcos\theta \dot{\theta}+ r_ccos(\theta +\phi)(\dot\theta +\dot\phi)\\
\end{align*}





\subsubsection{Lagrangian Mechanics}
Wit the velocity of the mass, the Lagrangian $L$ can be expressed as:
\begin{align*}
L &= T-V = \frac{1}{2}m(\dot x^2+\dot y^2) - V_{spring} - V_{gravity}\\
 &= \frac{1}{2}m(\dot l^2 + l^2\dot \theta^2 + r_c^2(\dot\theta+\dot\phi)^2) - \frac{1}{2}k(l-l_0)^2-mg(lsin\theta + r_csin(\theta +\phi))
\end{align*}
\begin{align*}
\frac{\partial L}{\partial l} &= -mgsin\theta+ml\dot\theta^2 - k(l-l_0)\\
\frac{\partial L}{\partial \dot l} &= m\dot l\\
\frac{d\partial L}{dt\partial \dot l} &= m\ddot l
\end{align*}


\noindent
Take $l$, $\theta$, $\phi$ as the generalized coordinate, the equation of motions are:
\begin{align*}
m\ddot{l} - ml^2\dot{\theta}^2 + k(l-l_0) &= -mgsin\theta\\
ml^2\ddot{\theta} + 2ml\dot l \dot{\theta} + mr_c^2(\ddot\theta+\ddot\phi) &= -mglcos\theta - mgr_c(cos(\theta + \phi))\\
mr_c^2(\ddot\theta+\ddot\phi) &= - mgr_c(cos(\theta + \phi))
\end{align*}
\noindent
However, this will not work, because the lack of the mass for the first link, which will cause the inertia matrix singular.
\pagebreak

\subsection{System Kinematics}
As indicated in Fig XXX, the position 
and the velocity of the frame $m$ are:
\begin{align*}
x &= lcos\theta \\
z &= lsin\theta \\
\dot x &= \dot lcos\theta -lsin\theta\dot \theta \\
\dot z &= \dot lsin\theta + lcos\theta \dot{\theta}
\end{align*}
The position 
and the velocity of the body $m_b$ are:
\begin{align*}
x_b &= lcos\theta + r_ccos(\theta +\phi)\\
z_b &= lsin\theta + r_csin(\theta +\phi)\\
\dot x_b &= \dot lcos\theta -lsin\theta\dot \theta -  r_csin(\theta +\phi)(\dot\theta +\dot\phi)\\
\dot z_b &= \dot lsin\theta + lcos\theta \dot{\theta}+ r_ccos(\theta +\phi)(\dot\theta +\dot\phi)\
\end{align*}





\subsubsection{Lagrangian Mechanics}
Wit the velocity of the masses, the Lagrangian $L$ can be expressed as:
\begin{align*}
L &= T-V = \frac{1}{2}m(\dot x^2+\dot z^2) + \frac{1}{2}m_b(\dot x_b^2+\dot z_b^2) - V_{spring} - V_{gravity}- V_{b_{gravity}}\\
 &= \frac{1}{2}m(\dot l^2 + l^2\dot \theta^2) +  
 \frac{1}{2}m_b(\dot l^2 + l^2\dot \theta^2 + r_c^2(\dot\theta+\dot\phi)^2) - \frac{1}{2}k(l-l_0)^2-mg(lsin\theta)-m_bg(lsin\theta + r_csin(\theta +\phi))
\end{align*}
\noindent \underline{EOM of $l$:}
\begin{align*}
\frac{\partial L}{\partial l} &= -(m+m_b)gsin\theta+(m+m_b)l\dot\theta^2 - k(l-l_0)\\
\frac{\partial L}{\partial \dot l} &= (m+m_b)\dot l\\
\frac{d\partial L}{dt\partial \dot l} &= (m+m_b)\ddot l
\end{align*}
\noindent \underline{EOM of $\theta$:}
\begin{align*}
\frac{\partial L}{\partial \theta} &= -(m+m_b)glcos\theta-m_bgr_ccos(\theta+\phi)\\
\frac{\partial L}{\partial \dot \theta} &= (m+m_b)l^2\dot{\theta} + m_br_c^2(\dot{\theta} + \dot{\phi}^2)\\
\frac{d\partial L}{dt\partial \dot \theta} &= (m+m_b)l^2\ddot{\theta} + 2(m+m_b)l\dot l\dot \theta + m_br_c^2(\ddot{\theta}+\ddot{\phi}) 
\end{align*}
\noindent \underline{EOM of $\phi$:}
\begin{align*}
\frac{\partial L}{\partial \phi} &=-m_bgr_ccos(\theta+\phi)\\
\frac{\partial L}{\partial \dot \phi} &= m_br_c^2(\dot{\theta} + \dot{\phi})\\
\frac{d\partial L}{dt\partial \dot \phi} &= m_br_c^2(\ddot{\theta}+\ddot{\phi}) 
\end{align*}
\noindent
Take $l$, $\theta$, $\phi$ as the generalized coordinate, the equation of motions are:
\begin{align*}
(m+m_b)\ddot{l} - (m+m_b)l^2\dot{\theta}^2 + k(l-l_0) &= -(m+m_b)gsin\theta\\
(m+m_b)l^2\ddot{\theta} + 2(m+m_b)l\dot l \dot{\theta} + m_br_c^2(\ddot\theta+\ddot\phi) &= -(m+m_b)glcos\theta - m_bgr_c(cos(\theta + \phi))\\
m_br_c^2(\ddot\theta+\ddot\phi) &= - m_bgr_c(cos(\theta + \phi))
\end{align*}
\noindent
\pagebreak