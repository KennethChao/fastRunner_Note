\section{Code implementation}





\subsection{Modeling and Parameters}
Main idea: a virtual wheel (as the massless leg) with radius $r_{wheel}$ penetrate the ground for a distance $r_{pen}$ where a external force point $pe$ is attached on it. A body (with mass $m$ and inertia $Iyy$) is attached to the center of wheel. 
Using PD control to interpret contact force when $p_e$ is under the ground.

\subsubsection*{06/07 First prototype (Not used now)}
\begin{itemize}
\item Joint numbers: 2
\item Joint types: Floating planer joint for virtual wheel and pin joint for the body link.
\item Contact point type: External force point
\item Virtual wheel rotation: set proper initial condition for virtual wheel (also need a large inertia to make it nearly constant).
\end{itemize}

Contact force: Assuming the ground height is $0$,
\begin{align}
F_z &= kp(0-pe_z) + kd(0 - ve_z)\\
\phi &= atan2(pe_x,r_{wheel}-pe_z)\\
F_x &= F_ztan(\phi)
\end{align}
where $ve$ is the velocity vector of the contact point $pe$, $kp$ and $kd$ are the PD control parameters. $F_x$ is calculated so that the vector of ground reaction force $[F_x,F_y,F_z]^T$ will point towards the virtual pivot (the center of the virtual wheel).\\

Assessments:
\begin{itemize}
\item Need to set a non-zero inertia of massless virtual wheel (for numerical stability), otherwise the simulation will diverge.
\item The inertia of virtual wheel need to be a large one for constant rotational speed.
\item Suggestions: remove the massless link, attach the external force point to the body and change its position in the controller every time step.
\end{itemize}

\subsubsection*{06/08 Round Runner}
\begin{itemize}
\item Joint numbers: 1
\item Joint types: Floating planer joint for the body link.
\item Contact point type: External force point
\item Virtual wheel rotation: Assigning the external force point location with respect to the joint in an open loop manner.
\item 
Contact force: Assuming the ground height is $0$,
\begin{align}
F_z &= kp(0-pe_z) + kd(0 - ve_z)\\
\phi &= atan2(pe_x,r_{wheel}-pe_z)\\
F_x &= F_ztan(\phi)
\end{align}
where $ve$ is the velocity vector of the contact point $pe$, $kp$ and $kd$ are the PD control parameters. $F_x$ is calculated so that the vector of ground reaction force $[F_x,F_y,F_z]^T$ will point towards the virtual pivot (the center of the virtual wheel).\\
\end{itemize}



Assessments:
\begin{itemize}
\item The ground reaction force looks better, while the energy is not balanced (after a while it will move towards the negative $x$ direction)
\item The inertia of virtual wheel need to be a large one for constant rotational speed.
\item Suggestions: Use the ground contact point (instead of external force point) to see how it goes.

\end{itemize}

\subsubsection*{06/11 Round Runner(with Ground Contact Point)}
\begin{itemize}
\item Joint numbers: 1
\item Joint types: Floating planer joint for the body link.
\item Contact point type: Ground contact point, linear contact model\footnote[1]{Disable the hardening stiffness in z direction by setting groundStiffeningLength to Double.NEGATIVE\_INFINITY}
\item Virtual wheel rotation: Assigning the external force point location with respect to the joint in an open loop manner.
\item \textbf{Contact point number} Parameterized, currently set to 3-6 points.
\item
Contact force: using built-in functionalities, only assigning the $kp$, $kd$ (PD parameters in the z direction), $kp_x$, and $kd_x$ (PD parameters in the x/y directions).
\end{itemize}



Assessments:
\begin{itemize}
\item Was able to generate a stable walking. Contact point has sliding.
\item Due to setting up stiffness and damping for $x$ and $z$ separately, the force is not always point towards the virtual pivot.

\end{itemize}

\subsubsection*{06/12 Round Runner(with External Contact Point Point)}

\begin{itemize}
\item Implement the same one as 06/11, but replace the ground contact point to the external one (because it is more complex for ground contact point to adjust stiffness/damping as parameters.)
\item implement the linear ground contact model basically.
\end{itemize}


\subsubsection*{06/13 Round Runner}

\begin{itemize}
\item Parameterize contact point numbers
\item Adding enum for switching between different setup: contact point type and the corresponding ground reaction force calculation: (w.r.t to the world frame or inertia frame.)
\end{itemize}
\subsubsection*{06/16 Round Runner (vertical hopper)}

\begin{itemize}
\item Adding vertical hopper with open-loop force control
\item Playing with open-loop force magnitudes for different stability conditions
\end{itemize}
\pagebreak