\documentclass{article}
\usepackage[utf8]{inputenc}
\usepackage{amsmath, amssymb}
\usepackage[margin=1.0in]{geometry}
\usepackage{graphicx}
\usepackage{mathrsfs}
\usepackage{hyperref}

\title{Note Fast Runner}
\author{Ken}
\date{June, 2018}

\usepackage{natbib}
\usepackage{graphicx}
\usepackage{amsthm}
 

\newtheorem{prop}{Proposition}

\begin{document}

\maketitle
\section{System dynamics}

At the end of touchdown, assuming impact force is $\sigma$.

\begin{align}
\label{eqn:SLIP1}
I \ddot{(\theta)} = \sigma  x_z
\end{align}
\[
y={C}x+{D}u\]
%where x is a vector with n states (i.e. $x\in R^n$) called state variables, u is the control input, y the is control output, and A is a n by n matrix. Depends on the dimension of input and output, the system can be classified as single-input single-output (SISO) system or multi-input multi-output (MIMO) system.\\\\
%Other notes:
% \begin{itemize}
% \item For most of the systems that follow causality, the output has no instantaneous input (i.e. $D=[0]$).
% \item Time-invariant system (don't need to be linear) is also called \emph{autonomous system}. On the other hand, time-varying system is  called \emph{non-autonomous system}.
%\item When B=[0], the system is also called a \emph {unforced system}, as show in Eq \eqref{eqn:unforce}:
%\begin{align}
%\label{eqn:unforce}
%\dot {x}={A}x,
%\end{align}
%\end{itemize} 
%%
%\section{Solution for LTI system}
%\subsection{Laplace transform of a LTI system}
%Assume the initial condition of the state variable $x$ is $x_0$. By applying Laplace transform on both sides of Eq \eqref{eqn:unforce}:
%\[
%sX(s)-x_0=AX(s)\]
%\[\rightarrow(SI-A)X(s)=x_0\]
%\begin{align}
%\label{eqn:LA}
%\rightarrow X(s)=(SI-A)^{-1}x_0,
%\end{align}
%where $I$ is identity matrix. We can derive the Laplace transform  of $x$ : $X(s)$ as shown in Eq\eqref{eqn:LA}. By applying inverse Laplace transform on both sides of Eq\eqref{eqn:LA}, then the solution of LTI system can be derived as follows:
%\[
%\mathscr{L}^{-1}(X(s))=\mathscr{L}^{-1}((SI-A)^{-1}x_0)\]
%\begin{align}
%\label{eqn:sol}
%\rightarrow x(t)=\mathscr{L}^{-1}(\underbrace{(SI-A)^{-1}}_\text{resolvent matrix})x_0=e^{At}x_0
%\end{align}
%
%\subsection{Matrix exponential}
%For a arbitrary constant $a\in R$, $\mathscr{L}^{-1}(s-a)=e^{at}$. Similarly, if $A$ is a square matrix, $\mathscr{L}^{-1}(sI-A)=e^{At}$. The next coming question will be how to evaluate the \emph {matrix exponential} $e^{At}$? One way to go is using Taylor series expansion of the exponential function, which is shown as follows:\\\\
%For $a\in R$ is a constant,
%\[e^{at}=1+at+\frac{1}{2!}a^2t^2+\frac{1}{3!}a^3t^3+\ldots=
%\sum\limits_{k=1}^\infty \frac{1}{k!}a^kt^k\]
%For $A$ is a n by n matrix,
%\[e^{At}=I+At+\frac{1}{2!}A^2t^2+\frac{1}{3!}A^3t^3+\ldots=
%\sum\limits_{k=1}^\infty \frac{1}{k!}A^kt^k\]
%The Taylor series expansion of matrix exponential can also be used for verifying the solution of x:
%\[\dot x=\frac{dx}{dt}=\frac{d}{dt}e^{At}x_0\]
%\[\frac{d}{dt}e^{At}x_0=\frac{d}{dt}(I+At+\frac{1}{2!}A^2t^2+\frac{1}{3!}A^3t^3+\ldots)x_0\]
%\[\frac{d}{dt}e^{At}x_0=(A+A^2t+\frac{1}{2!}A^3t^2+\ldots)x_0\]
%\[\frac{d}{dt}e^{At}x_0=A(I+At+\frac{1}{2!}A^2t^2+\frac{1}{3!}A^3t^3+\ldots)x_0=Ax\]
%Matrix exponential $\Phi(0,t)=e^{At}$ is also called the \emph{state-transition matrix}\\
%(Please refer to: \url{https://en.wikipedia.org/wiki/State-transition_matrix} )
%
%\subsection{Solution of LTI system with control input}
%Considering the following derivative equation:
%\[\frac{d}{dt}(e^{-At}x(t))=-Ae^{-At}x+e^{-At}\dot x\]
%\[=e^{-At}(\dot x-Ax)\]
%\[=e^{-At}(Bu)\]
%Integrate the left hand side from $0$ to $t$:
%\[\int_{0}^{t} \frac{d}{dt}e^{-At}x(t)=[e^{-A\tau}x]_0^t=e^{-At}x(t)-x_0=\int_0^te^{-A\tau}Bu(\tau)d\tau\]
%\[\rightarrow e^{-At}x(t)=x_0+\int_0^te^{-A\tau}Bu(\tau)d\tau\]
%\[\rightarrow x(t)=\underbrace{e^{At}x_0}_\text{homogeneous solution}+\underbrace{\int_0^te^{-A(t-\tau)}Bu(\tau)d\tau}_\text{particular solution}\]
%
%
%\section{Eigenvalues, eigenvectors and system stability}
%\subsection{Eigenvalues and eigenvectors}
%For a matrix $A$, its eigenvalues and eigenvectors can be expressed as follows:
%\begin{align}
%\label{eqn:eigen}
%Av=\lambda v
%\end{align}
%where $\lambda$ is the eigenvalue (a scalar), and v is the eigenvector.
%If $A$ matrix is a n by n full rank matrix, then it will have n eigenvalues with n corresponding eigenvectors.\\\\
%If matrix $A$ is related to the dynamics of a system (e.g. A matrix in Eq \eqref{eqn:LTI}), its eigenvalue and eigenvector usually contain important physical meanings.\\\\
%One interpretation of Eq \eqref{eqn:eigen}: on a specific direction along the eigenvector v, one can use a \emph{scalar} $\lambda$, instead of the system matrix, to describe the system dynamics.
%
%\subsection{Cayley–Hamilton theorem}
%To derive the eigenvalues of a n by n matrix A, one way is to solve the characteristic equation (characteristic polynomial), which is defined as:
%\[det(\lambda I-A)=\lambda^n+a_{n-1}\lambda^{n-1}+\ldots+a_1\lambda+a_0=f(\lambda)=0, \]
%where $I$ is a n by n identity matrix.\\\\
%Cayley–Hamilton theorem states that substituting matrix $A$ for $\lambda$ in the characteristic equation results in the zero matrix, that is,
%\begin{align}
%\label{eqn:CH}
%f(A)=A^n+a_{n-1}A^{n-1}+\ldots+a_1A+a_0I=0
%\end{align}
%By using the equation above, one can utilize it to calculate $A^n$ (i.e. the power of $A$) with reduction of order, or calculate the inverse of matrix $A$ by multiplying $A^{-1}$ on both sides.
%
%\subsection{Similar Transform (Similarity transformation)}
%Definition:  matrices $A$,$B$ $\in R^{n\times n}$, $\exists$ (there exists) $P$ $\in R^{n\times n}$ which is invertible s.t. (such that) $B=PAP^{-1}$\\\\
%Remarks of similar transform
% \begin{itemize}
% \item The eigenvalues of matrices $A$ and $B$ will be the same (i.e. system characteristic is preserved!)
% \item The characteristic equations, ranks, determinants, traces of $A$ and $B$ will also be the same.
% \end{itemize}
%Example:
%A is a n by n matrix with distinct eigenvalues ($\lambda_1\ldots\lambda_n$) and eigenvectors ($v_1\ldots v_n$), then the following equations are satisfied:
%\[Av_1=\lambda_1v_1\]
%\[Av_2=\lambda_2v_2\]
%\[\vdots\]
%\[Av_n=\lambda_1v_n\]
%Those equations can be expressed as the following:
%\begin{align}
%\label{eqn:ST}
%AV=V\Lambda
%\end{align}
%where
%\[V=[v_1,v_2,\ldots,v_n],\]
%\[\Lambda=\left[ \begin{matrix}
%   \lambda_1 &0 &\ldots &\ldots&0 \\
%   0&\lambda_2 & 0&\ldots &0\\
%   \vdots&&\ddots&&\vdots\\
%   0&\ldots&\ldots&0&\lambda_n \end{matrix} \right],\]
%Multiply $V^{-1}$ to the right on both side of Eq \eqref{eqn:ST}, then we can show $\Lambda$ (or the \emph{Jordan form}) is a similar transform of A:
%\[A=V\Lambda V^{-1}\]
%
%\subsection{System stability}
%For a linear system, $\dot x=Ax$ is stable if $Re[\lambda (A)]<0$. In other words, the system is stable if the 
%\emph(zeros) of characteristic equation (i.e. the \emph{poles} of the system) are in the left half plane.
%The criteria above can be used to identify the stability of any linear system. However, this criteria can only works partially for the nonlinear systems. In other words, a more general definition of stability is required for nonlinear system analysis. As a result, Lyapunov proposed other definitions of stability:
%\subsubsection{Lyapunov stability}
%Given a system $\dot x=Ax, x(0)=x_0$, the system is stable if $||x(t)||^2$ is monotonically decrease (i.e. non-increase). Therefore, the condition of Lyapunov stability can be stated as: the system is asymptotically stable if\\
%\[\frac{d}{dt}||x(t)||^2<0\]
%The $||x(t)||^2$ can be expressed as a inner product of vectors (assume x is a colume vector):
%\[\rightarrow \frac{d}{dt}||x(t)||^2=\frac{d}{dt}x^Tx=\dot{x}^Tx+x^T\dot{x}\]
%\[=\dot{x}^TA^Tx+x^TAx\]
%\[=\dot{x}^T(A^T+A)x<0\]
%Example of a stable system: 
%If \[(A^T+A)\leq-vI,\] 
%\[v\geq 0,\]
%where $v \in R$, then the system is asymptotically stable.
%Note:
% \begin{itemize}
% \item Asymptotic stability, which means that the system state will converge to the equilibrium point when $t\rightarrow \infty$
% \item Exponential stability (stronger version) is a form of asymptotic stability, which indicates that the system's convergence will be bounded by a \emph{exponential decay}. Compared to asymptotic stability, exponential stability indicate the minimum rate of decay of the system state response.
% \item For linear systems, all stable systems are guaranteed to be exponentially stable.
% \end{itemize}
%\subsubsection{Quadratic Lyapunov Function}
%Definition of positive definite:
%Given a matrix $P$, P is positive definite (denoted as $P>0$) if $x^TPx>0$ for any $x$ vector except $x$ is a zero vector. Similarly, P is negative definite ($P<0$) if $x^TPx<0$ for any $x$ vector except $x$ is a zero vector.\\
%A more formal version of Lyapunov stability is stated as shown below
%\[x^TPx, P>0,\]
%\[\frac{d}{dt}x^TPx=\dot{x}^TPx+x^TP\dot{x}+x^T\dot{P}x\]
%If system is LTI, then $\dot P=[0]$.
%\[\rightarrow\frac{d}{dt}x^TPx=x^T(A^TP+PA)x\]
%The system is asymptotically stable if the matrix $(A^TP+PA)<0$. \\\\
%Theorem of Lyapunov function (for continuous linear system):
%Given any matrix $Q>0$, $\exists$ unique $P>0$ satisfying $A^TP+PA=-Q$ if and only if the linear system $\dot{x}=Ax$ is globally asymptotically stable. The function $V(x)=x^TPx$ is a Lyapunov function can be used to verify system stability. In addition, the closed form solution of P is availablea as shown:
%\[P=\int_0^\infty e^{A^Tt}Qe^{At}dt\]
%
%\section{Controllability and Observability}
%A system is controllable on $[0,t]$ if given any initial condition, $\exists$ a continuous input $u(t)$ such that $x(t)=0$
%\subsection{Controllability, Controllability Gramian and Controllability Matrix}
%The controllability Gramian is a Gramian used to determine whether or not a linear system is controllable,which is defined as:
%\begin{align}
%\label{eqn:gc}
%\omega(t,0)=\int_0^t\Phi (\tau,0)BB^T\Phi^T(\tau,0)d\tau
%\end{align}
%
%\begin{prop}
%The system is controllable if and only if $\omega(t,0)$ is invertible.
%\end{prop}
%\begin{proof}[Proof of sufficiency]
%If we assume $omega$ is invertible, then we can always define the input as
%\[u=-B^T\Phi^T\omega^{-1}\Phi x_0,\] such that
%\[x(t)=\Phi x_0-\int_0^t\Phi BB^T\Phi^Tw^{-1}\Phi x_0=\Phi x_0-\omega\omega^{-1}\Phi x_0=0\]
%\end{proof}
%
%\begin{prop}
%The system is controllable if and only if the controllability matrix $C_{mat}= \left[ \begin{array}{ccccc} B & AB & A^B & \dots & A^{n-1}B \end{array} \right]$ is full rank.
%\end{prop}
%\begin{proof}[Proof of necessity]
%the controllability Gramian $\omega=\int_0^t\Phi BB^T\Phi d\tau$ can be viewed as the inner product of $\Phi B$: $<\Phi B,\Phi B>=||\Phi B||^2$. Assume $\omega$ is not invertable (i.e. singular), then 
%\[\exists X_a\neq0\] such that \[\omega X_a=0,\]
%where the following equations can thus be derived:
%\[\rightarrow {X_a}^T\omega{X_a}=0\]
%\[\rightarrow ||{X_a}^T\Phi B||=0\]
%\[\rightarrow {X_a}^T\Phi B=0\]
%\begin{align}
%\label{eqn:CM}
%\rightarrow {X_a}^Te^{At} B=0
%\end{align}
%Differentiate Eq \eqref{eqn:CM} about $t$ for $n-1$ times:
%\[{X_a}^Te^{At} AB=0\]
%\[{X_a}^Te^{At} A^2B=0\]
%\[\vdots\]
%\[{X_a}^Te^{At} A^{n-1}B=0\]
%\[\rightarrow {X_a}^Te^{At}\left[ \begin{array}{ccccc} B & AB & A^B & \dots & A^{n-1}B \end{array} \right]={X_a}^Te^{At}C_{mat}=0,\]
%thus $C_{mat}$ is also not invertible(i.e. singular)
%\end{proof}
%\subsection{Observability}
%Observability is a measure for how well internal states of a system can be inferred by knowledge of its external outputs (i.e. If you can measure the output, then you can use it to derive the initial condition of state variables). The observability and controllability of a system are mathematical duals.
%\begin{prop}
%The system is observable if and only if the observability matrix $O_{mat}= \left[ \begin{array}{ccccc} C \\ CA \\ CA^2 \\ \vdots \\CA^{n-1} \end{array} \right]$ is full rank.
%\end{prop}
%\section{Conclusion}
%``I always thought something was %fundamentally wrong with the %universe'' \citep{adams1995hitchhiker}

%\begin{figure}[h!]
%\centering
%\includegraphics[scale=1.7]{universe.jpg}
%\caption{The Universe}
%\label{fig:univerise}
%\end{figure}



\bibliographystyle{plain}
%\bibliography{references}
\end{document}
